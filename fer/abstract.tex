\thispagestyle{empty}
\addcontentsline{toc}{chapter}{Abstract}
\begin{center}
  \begin{tabular}{p{13cm}}
    \vspace{6ex}

    \begin{center}
      \textbf{\Large \thesisTitle}\\
      \vspace{2ex}
      \textbf{\large Abstract}\\
    \end{center}
    \vspace{2ex}

    The focus of this project is enhancement of functionality, reliability and sensor accuracy of an intelligent bed that monitors human sleep. The scope of the project includes the implementation of an application layer protocol and network communication between the embedded system in the bed and remote server. Furthermore, the implementation possibilities of data preprocessing, filtering, and automatic sleep analysis are explored and tested. The system is tested and evaluated in the Ubiquitous Computing Laboratory at the Hochschule Konstanz University of Applied Sciences.\\
    \vspace{2ex}
    \textbf{Keywords:} sleep tracking, embedded systems, sensor meshes, sleep analysis

    \vspace{15ex}

    \begin{center}
      \textbf{\Large \thesisTitleHr}\\
      \vspace{2ex}
      \textbf{\large Sa\v{z}etak}\\
    \end{center}
    \vspace{2ex}

    Tema projekta je unaprje\dj{}enje funkcionalnosti, pouzdanosti i preciznosti rada inteligentnog kreveta koji prati ljudski san. U sklopu projekta implementira se aplikacijski sloj te ostvaruje mre\v{z}na komunikacija izme\dj{}u ugradbenog sustava u krevetu i udaljenog ra\v{c}unalnog servera. Nadalje, rad istra\v{z}uje i testira implementaciju preprocesiranja podataka, izrade podatkovnih filtera i automatske obrade i analize podataka o snu. Sustav se testira i evaluira u Laboratoriju za sveprisutno ra\v{c}unarstvo pri Hochschule Konstanz University of Applied Sciences.\\
    \vspace{2ex}
    \textbf{Klju\v{c}ne rije\v{c}i:} pra\'{c}enje sna, ugradbeni sustavi, mre\v{z}e senzora, analiza sna
    
	\end{tabular}
\end{center}