\chapter{Introduction}
To help readers get better acquainted with the topic, introduction is divided into four sections. First section covers general motivation and relevance of the project. Next section describes state of technology, market and consumer trends at the time thesis was published. The following section lays out project goals and defines the scope they will be tackled on while in the last section, project structure is outlined so that readers can easily navigate through this thesis body.

\section{Motivation}
Sleep is seemingly a trivial thing - from the moment that they are born, all humans have a need to sleep. It is a natural function in the same way breathing and other vital body functions are. Having slept for adequate time and with good quality tends to make people feel good and have more energy performing their daily tasks. When a person did not sleep well or did not sleep enough it will usually negatively reflect both on their body and their behavior. National Sleep Foundation along with multi-disciplinary expert panel recommends sleep time for each age group ranging from 14 to 17 hours daily for newborns to between 7 and 8 hours for older adults\cite{NSF}. Sleep deprivation effects motor and cognitive abilities as well as mood but these effects can also occur in cases of bad sleep quality regardless of the sleep duration\cite{doi:10.1093/sleep/19.4.318}. That same sleep quality is influenced by many factors ranging from physical ones such as sleeping environment and position to subjective ones such as emotional state and dreams. As clear separation of these factors is rarely possible, most of the researches relied on the isolation of influences comparing results between large control and influenced groups. Sleep quality is then usually determined by questionnaires and data analysis which resulted in quantitative results such as Pittsburgh Sleep Quality Index\cite{psqi}.\\

A more technical way of determining the sleep quality is active sleep monitoring and tracking. This method involves continuous or periodical measurement of physical parameters. Some electrophysiological measurements that can be done to precisely parametrize sleep are \ac{EEG}, \ac{EMG} and \ac{EOG}. Drawback of using these methods is requirement of complex equipment, knowledge to evaluate the results and controlled environment which is why these measurements are usually done only for clinical or research purposes. But simple sleep monitoring can be done using much simpler processes - with heart rate, body movement and position tracking. Unlike before mentioned measurements these can be done unobtrusively and in home environment. Improving the process and accuracy of these methods as well as with correlation of collected results to the real sleep parameters may lead to much easier diagnosis of sleep disorders. <Describe benefits of better measurements and sleep tracking>\\

This thesis will primarily focus on finding a non-obtrusive way to track both sleep time and quality with proposal of technology and measuring methods.\\

<Propose few use cases for sleep tracking>

\section{Technology, market and industry trends}
<Describe current sleep tracking technology>
<Review market>
<Analyze current trends in health tracking>
At the time this thesis was published a trend

\section{Goal and scope}
<Describe product>
<Describe how the product will be used>

\section{Project outline}
TODO