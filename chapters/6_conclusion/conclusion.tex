\chapter{Conclusion}
\label{chap:conclusion}

Based on previous work and research it is concluded that it is possible to track subject position during sleep and to monitor vital signs using a grid of \ac{FSR} pressure sensors underneath the bed. Based on these unobtrusive measurements it is possible to classify sleep stages and measure sleep quality. This information is not only to useful for medical and research purposes but there is also a growing market share of people interested in sleep tracking who would like to self-improve their sleep quality. System architecture and implementation proposed in this thesis provide means to develop a robust and scalable product for sleep tracking. Hardware architecture was improved on the old one by reducing the number of nodes and by introduction of custom \ac{PCB}s that eliminate the need for additional hand-soldered boards. A communication inside of the system was also improved by introduction of system-wide bus. Endpoint node was reimagined to implement data storage and display features so that the system doesn't require additional servers and processing units. But, if multiple sensor grid systems are to be integrated into a larger one such as in case of a medical facility, where multiple patient vitals signs are tracked at the same time, an \ac{API} was developed to allow for this possibility. On the other hand, the developed webpage based user interface allows single users to interface the system and review their sleep without the need of external devices. As all of the technology used in this project is based on open source it is easy to grow and improve the system further.

The current system acquires, distributes and displays data but does not provide an insight into the sleep quality. Next steps to improve the system would be integration of position classification and body movement during sleep. This data could be paired with an output of the algorithm for vital sign detection that also gathers data from the pressure sensors. Together, these algorithms could possibly provide a quite accurate estimation of sleep stage and sleep quality. Moreover, they could be used in clinical purposes to monitor vital signs and help medical personnel prevent bed sores in bed-bound patients. For home use, implementation of sleep classification and automatic analysis algorithms could help end users sleep better. It would also be possible to integrate the system as a part of \ac{IoT} infrastructure for home automation.