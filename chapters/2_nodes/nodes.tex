\chapter{Sensor nodes}
\label{chap:nodes}

This chapter will describe functionalities of sensor node devices. After that it will describe how electronic parts were chosen. A separate section will include details regarding \ac{PCB} design while another section will follow the production from first prototype to the production. Finally, it will be shown how the board is installed and integrated into the system.

\section{Physical design and connections}

Most of the required \ac{PCB} features were already described in the previous chapter - board should allow direct connection of \ac{FSR} sensors, it should eliminate need for additional perforated boards and it should provide a more robust solution for physical board position detection. Furthermore, it should feature low power consumption and enable connection of at least two rows of 8 pressure sensors. Also it should feature small dimensions because and provide mounting holes so that it doesn't have to be suspended by wires or be taped to the bed. The ideal position for the installation of the board is under the bed base slates. This way it can easily be serviced.

There are 4 variants of pin endings found on \ac{FSR} sensors. A variant with no leads is used for custom pin endings while solder tabs variant is used for direct soldering to the board. Because of the materials used for construction of the sensor, when heat is applied using standard soldering iron, there is a high possibility of sensor leads melting. This is why a female plug connector option was chosen. Distance between leads is $1/10"$ and they are compatible with standard \ac{PCB}connector pins. Since sensors will be mounted on pressure disks which are found on the upper side of the bed base, while the board is found under the slates, elongation cables are required. In this case, DuPont "jumper" cables will be used and a board connectors have been designed in such a way. Two edges of the board, are populated with 8 two-row connectors. Row on the top is connected to the \ac{ADC} pins of the microcontroller while the bottom pins are grounded. In front of each of the pin, a reference resistor $R_ref$ is found.

Connector that was used for internal communication bus features 6 wires - 2 wires are used for \ac{I2C}, 2 are used for power supply and 2 are used for physical position recognition. Because they are interchangeable with DuPont jumper wires and because of toolless cable connector installation, \ac{IDC} cables and connectors were used\cite{IDC}. There are two 3x2 internal communication bus connectors on the board so that boards can easily be "daisy-chained". Close to the first connector, two pull-down resistors for I2C bus are situated. For microcontroller programming, the same type of connectors was used. The cortex debug interface consists of 10 pins so a single 5x2 connector was used.

To power the microcontroller and because of debugging possibilities, a type B micro \ac{USB} connector was also added to the board. \ac{USB} connection features 2 differential data signals, a $5V$ and ground. Connector additionally has a "on-the-go" identification pin which is left floating because the requirement does not require a board to become \ac{USB} host. What board implements is a PRTR5V0U2X \ac{ESD} diode\cite{PRTR} which helps protect both the host \ac{PC} and the \ac{PCB} from electrical stress in from of surge or overvoltage.


\section{Component selection and compatibility}

A most important part of the node circuitboard is microcontroller. Main requirement for it was the possibility of connecting 16 \ac{ADC} devices without the use of additional \ac{ADC} chips. 


\section{PCB design}



\section{Software implementation}



\section{Installation and integration}

