\chapter{Endpoint node}
\label{chap:endpoint}

Endpoint is a device that acts as an interface between sensor node network and external clients and services. This chapter focuses on the embedded system implementation, on how communication with nodes has been implemented and then on data acquisition process.


\section{System setup}

For an endpoint Intel Edison \ac{COM} was used. It runs x86 instruction set\cite{Edison} which should allow for compatibility with most of the embedded operating systems. Unfortunately, it has a custom architecture with integrated Intel Quark microcontroller as a coprocessor which most of the standard operating systems don't support. Intel in its \ac{BSP} provides a Yocto Linux image and compilation tools. For an easy setup, already compiled system images are available. Although Yocto is relatively easy to use, it requires an image to be compiled and uploaded for every change. This takes a lot of time in development process. Prebuilt system image provides most of the features needed by this project but it doesn't have a \ac{HTTP} server such as Nginx or Apache2. Instead, most of Edison projects are running \ac{HTTP} server inbuilt into NodeJS. This usually works just fine with simple applications providing data to 1 client. But if data is to be accessed by multiple clients, client requests are queued and served by a single core. Also, if the server crashes, there is no automatic restart or repair features. That's why port of Debian Linux distribution called Ubilinux was used in this project. For an \ac{HTTP} server Nginx was used. It was configured to monitor and automatically restart the application if it crashes. Another feature that nginx provides is automatic serving and caching of static files. Server was instructed to serve all javascript, css files and images directly from filesystem. Another feature that Nginx allows is logging so access and server log hold the data on the server usage. 

One of currently most popular programming languages is Python. It is a scripting language which supports object oriented programming and has some functional programming features. There are multiple frameworks designed to allow Python users to serve web pages. Mos popular ones are Django, Pyramid and Flask. Django is a big framework which is very structured and pragmatical which is a very good feature for big projects but doesn't work well for smaller ones. Pyramid on the other hand is very customizable but requires a lot of configuration to work as intended. Flask is actually a microframework which urges users to follow good design practices but does not require a lot of configuration or boiler-plate tasks to be done. Also, it is very easy to learn and use so it was a framework of choice for this project. Web pages and services programmed in Flask are usually served by uwsgi server application. uwsgi was configured in such a way that it serves data to Nginx through Linux socket which then serves web pages and content to the end user.

Since endpoint is also used for data accumulation, a data is stored in database. Database of choice for this project is PostgreSQL. To allow easier communication with database a SQLAlchemy \ac{ORM} toolkit was used. It allowed direct mapping of Python classes to ER modelled database tables. It removed the need to write SQL code in the application. Instead of that, a session is created which is then queried for data and which allows easy adding, update and delete of database rows. When a database model is developed for the first time, a table can be created. But when a new column needs to be added to the table, it's quite impractical to delete table and create a new one manually. This is why Flask package flask-migrate is providing possibility of using database migrations. This package provide possibility of automatic database initialization and when changes to the model are done it can be run to generate migration scripts and upgrade the database. To start nginx and postgres, and uwsgi systemd scripts are used. Postgres and nginx scripts were provided with the respected packages but a custom script had to be written for uwsgi. The script creates the directory for socket file and starts the application. A detailed description how to set the project up is available on \href{github.com/Xenosb/thesis-edison}{GitHub} page.


\section{Communication with sensor network}




\section{Data acquisition routine}
