\chapter{Evolution of system design}
\label{chap:evolution}

To track sleep unobtrusively, multiple techniques and sensor types have been tested and evaluated by other researchers. Placing load cells under the bed supports allowed researchers to determine the precise time when subject fell asleep and when subject woke up\cite{load_cells}. Infrared camera recording of subjects sleeping allowed precise recognition of small movements even under the blanket \cite{video}. Another research group used \ac{POF} sensors to recognize breathing patterns and detect apnea\cite{optical}. The same results were also achieved using pneumatic pressure sensors placed in a sealed air-cushion under the mattress\cite{pneumatic}. In yet another research, a group of researchers conducted experiment in which they placed two $24 GHz$ radars under the bed and found out that it is even possible to accurately recognize hearth rate\cite{radar}.

But one of the recently most popular methods of unobtrusive sleep tracking is much simpler and more affordable than others. It is called pressure sensing and involves continuous measurement of pressure from under the subject. This thesis picks up on the work of Prof. Dr. Ralf Seepold, Ra\'ina Kuhn, Daniel Scherz and Maxime Guyot at \ac{UC-Lab}\cite{Kuhn}\cite{Guyot} who have successfully used pressure sensors to determine position, detect movement and track vital signs during sleep.


\section{Devices and technology}

To achieve a good sleep tracking from pressure readings under the bed, an appropriate environment has to be selected. Environment consists of an adequate bed frame, a mattress and of base-plates which hold the mattress in the frame of the bed. Bed frame is of a regular size - $90 * 200cm$ which accommodates vast majority of people. Because of good pressure propagation mattress should not be too firm or too tight. Therefore mattress of uniform hardness level 2 has been selected.

To hold the mattress in place, a grid of pressure-disks is used. They are critical part of the system as they have to absorb the pressure from the mattress. Also, they are a point where pressure measurement cane be done. To provide a better granularity, a pressure-disks provided by ErgoProTech and depicted in Figure \ref{fig:base-plate} was used. They were placed 3cm from each other and under the whole area of the mattress according to the manufacturers usage recommendation. This ensures adequate support across the mattress. For selected mattress size, base-plate grid consists of 12 rows and 5 columns totaling in 60 pressure-disks. Pressure-disks are made of semi-elastic polymer and consist of 4 connected rectangular pads. These pads are supported by 4 double arms anchored at the same point where the whole structure is connected to the bed frame. There are multiple hardness levels of base plates and they are differentiated by the color of the rectangular pads. Firmer base plates are grey while more elastic ones are purple. More about placement of pressure-disks can be found in \autoref{ssec:test_environment}

\begin{figure}[h]
  \begin{center}
    \includegraphics[width=0.5\linewidth]{1-base_plate.jpg}
  \end{center}
  \caption{Base plates in the bed.}
  \label{fig:base-plate}
\end{figure}

There are multiple types of pressure sensors that could be used for the purpose of this project\cite{pressure_sensors}. Potentiometric pressure sensors are very crude due to their construction and often have realiability and hysteresis issues. Inductive pressure sensors require \ac{AC} excitation of coils and consequentially signal filtering and demodulation. Piezoelectric and piezoresistive pressure sensors measure change of pressure using piezoelectric effect. Capacitive pressure sensors use a small diaphragm as a capacitor plate. When pressure is applied, diaphragm deflects and capacitance changes. Change may or may not be linear and usually is on the order of a few \ac{pF} while total capacitance of the sensor is between $50$ and $100 pF$. This means that it may be hard to precisely measure the values and this method may also suffer from environmental effects and \ac{PCB} or protoboard design. \ac{FSR} is a type of material whose resistance changes when a pressure is applied. Advantages of \ac{FSR} sensors over other pressure sensor are possibility of detecting static pressure, its flexibility, thinness and inexpensiveness. Multiple other researches were made used the same type of sensor and have proven its reliability and accuracy when used in a sensor grid\cite{pillow_system_1}\cite{pillow_system_2}\cite{apnea_low_cost}. Therefore, this type of sensors was selected for use in this specific project.

So how does the \ac{FSR} sensor work? \ac{FSR} is a \ac{PTF} device which exhibits a decrease in resistance with an increase of the force applied to the active surface\cite{fsr_guide}. At 'zero force', conductive ink is separated from the active area by spacer adhesive and in that case \ac{FSR} sensor has the highest possible resistance. When pressure is applied to sensor, conductive layer is pushed down on the active area which results in a decrease of resistance. Construction of FSR sensor is provided in \ref{fig:fsr-sensor}. For a hemispherical sensor with a diameter of 12.7mm (model 402) sensibility starts just below $20 g$ and extends to around $10 kg$ when saturation occurs. Passing a threshold at $20 g$, resistance changes from greater than $100 k\Omega$ to $10 k\Omega$. After that resistance falls logarithmically with an increase of force as seen in \ref{fig:fsr-sensor}. For consistent results application manual\cite{fsr_guide} suggests using a firm, flat and smooth mounting surface and use of a rubber spring to spread the pressure over the whole sensor. Also, an appropriate sensor size and shape is to be used.

\begin{figure}[h]
  \begin{center}
    \includegraphics[width=0.50\linewidth]{1-fsr_sensor.png}
    \includegraphics[width=0.33\linewidth]{1-fsr_graph.png}
  \end{center}
  \caption{FSR sensor construction and resistance characteristics.}
  \label{fig:fsr-sensor}
\end{figure}

There are 4 different types of standard of-the-shelf sensors\cite{fsr_guide} and their sizes and shapes are shown in Table \ref{tab:fsr_types}. Models 400 and 402 also have short-tailed variants which feature shorter connection between \ac{FSR} pads and pins found at the end of the lead wires. These same models have a small pressure sensing surface and are not as well suitable for this project. Models 406 and 408 have much larger contact surface area, which means that a more consistent distribution of pressure is possible. Model 406 is perfect for use in areas that require good position resolution such as scapular area. On the other hand, type 408 can be used in crural region.

\begin{table}[h]
  \begin{center}
    \begin{tabular}[h]{ | >{\centering\arraybackslash} m{3cm} | >{\centering\arraybackslash} m{3cm} | >{\centering\arraybackslash} m{7cm} | }
      \hline
      Part number & Description & Part image \\ 
      \hline
      Model 400 & 0,2" circle & \includegraphics[height=1.3cm]{1-fsr_400.png} \\  
      Model 402 & 0.5" circle & \includegraphics[height=1.3cm]{1-fsr_402.png} \\
      Model 406 & 1.5" square & \includegraphics[height=1.3cm]{1-fsr_406.png} \\  
      Model 408 & 24" strip & \includegraphics[height=1.3cm]{1-fsr_408.png} \\
      \hline
    \end{tabular}
  \end{center}
  \caption{Standard shapes and sizes of FSR sensors offered by Interlink Electronics.}
  \label{tab:fsr_types}
\end{table}

To get a reading of sensor resistance ($R_{fsr}$), a sensor is connected in a series with a fixed value reference resistor ($R_{ref}$). Then, an input voltage ($V$) is applied to the circuit. Voltage drop ($V_{fsr}$) is measured on the \ac{FSR} sensor leads. Pressure applied to sensor is in a reciprocal correlation to the $R_{fsr}$ because \ac{FSR} has a maximal resistance when there is no external force pressuring its surface as seen in Figure \ref{fig:fsr-sensor}. The same graph is also sampled for force-resistance pairs which are used for reference resistor selection. $R_{fsr}$ needs to be selected in such a way that it has best resolution for force between $0 kg$ and $1.6 kg$. These weight values were selected based on R. Kuhns calculation\cite{Kuhn}. She took an average weight of a person and mattress and calculated an estimate of how much weight each of the disk springs carry. Equation \ref{eq:fsr-calculation} describes the relation between $V_{fsr}$ and $R_{fsr}$ when a $R_{ref}$ has a fixed value. Multiple standard resistor values were put into the equation and at resistance of $10 k\Omega$ change gradient was highest. Therefore, $10 k\Omega$ resistor was used as $R_{ref}$.

\begin{figure}[h]
  \begin{equation}
    \label{eq:fsr-calculation}
    V_{fsr} = \frac{V*R_{ref}}{R_{fsr}+R_{ref}}
  \end{equation}
\end{figure}

Initial sensitivity tests that were conducted by R. Kuhn and M. Guyot showed that additional layer should be added on top of the \ac{FSR} sensors to help with pressure absorption. In Figure \ref{fig:fsr-sensor} it is clearly visible that adhesive spacer layer creates a non-sensitive frame around the active sensor area. When sensor surface was directly exposed to the mattress, adhesive absorbed most of the pressure as it was not as elastic as active area. This was solved by using felt\footnote{textile material that is produced by matting, condensing and pressing fibers together.} gliders. This greatly improved the sensitivity and the results can be seen in Figure \ref{fig:felt_gliders}.

\begin{figure}[h]
  \begin{center}
    \includegraphics[width=0.7\linewidth]{1-felt_gliders.jpg}
  \end{center}
  \caption{Comparison of pressure with and without felt gliders.}
  \label{fig:felt_gliders}
\end{figure}

To convert voltage to a digital value \ac{ADC} was done with a help of a microcontroller. Sensors were connected to the Trinket Pro 5V microcontroller development board\cite{Trinket}. This board features ATMega328P microcontroller with integrated \ac{ADC} functionality\cite{atmega328p}. From a myriad of different boards this one was chosen because it can be programmed as Arduino Pro Mini but features 8 analog input pins. Unfortunately, two of the analog pins share functionality with \ac{I2C} protocol and 1 was used for board identification. This means that 1 board could support up to 5 sensors. To get collect readings from multiple devices already mentioned \ac{I2C} protocol was used. A device that takes the readings from the pressure sensors and can communicate with the rest of the system will be called a node in the rest of the thesis.

But it would be quite impractical to connect a PC to each an every node to collect data so the system was designed with a new device as an endpoint. This device communicates as \ac{I2C} master device with the nodes and allows easier communication between user and nodes. For purpose of an endpoint, Intel Edison \ac{SOM} was used. It features Intel Atom \ac{CPU} and Intel Quark 32-bit microcontroller\cite{Edison}. Both have x86 architecture and use x86 instruction set. But what is more important, Intel Edison has $4GB$ \ac{EMMC} storage as well as integrated \textit{Bluetooth} and \textit{Wi-Fi}. This features allow storing larger amount of collected sensor readings and easier interaction with the system. A simple web application was developed in Node.js\ref{nodejs} and deployed on the Edison.

\section{Test environment}
\label{ssec:test_environment}

If a sensor grid would have 8 rows with 8 sensors, it would require at least 13 boards. As all of the boards were communicating using \ac{I2C} protocol, they required addresses that were in accordance with their physical position. This would either require a different firmware for every board or some smarter alternative in which every board would automatically select different I2C address in an orderly way. The solution that was proposed was using same-value resistors in a series. Depending on the voltage difference between adjacent board identification resistor and ground an \ac{I2C} address was chosen. Although this solution was easy to implement, it had used up one ADC pin that could have been used for sensor connection. Since Trinket Pro does not have a prototyping holes, 5 reference resistors for sensor reading and 1 for position identification had to be added. Reference resistors were soldered as a part of the wires and isolated using heat-shrink tube while position identification resistor was soldered onto a piece of perforated board and connected between two Trinkets.

If all of the 
An ideal sensor positioning and distribution would result in 100\% coverage of the sleeping area. If type 406 sensors would be used, this would require use of 1272 sensor units

\begin{figure}[h]
  \begin{center}
    \includegraphics[width=0.4\linewidth]{1-weight_distribution.png}
    \includegraphics[width=0.4\linewidth]{1-sensor_layout.jpg}
  \end{center}
  \caption{Sensor arrangement in bed compared to sleep position.}
  \label{fig:sensor-layout}
\end{figure}

\begin{figure}[h]
  \begin{center}
    \includegraphics[width=0.4\linewidth]{1-sleep_positions.jpg}
  \end{center}
  \caption{Different sleep positions.}
  \label{fig:sleep_positions}
\end{figure}


\section{Results so far}

\section{A new architecture}
